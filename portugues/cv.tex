\documentclass[10pt, a4paper]{article}

%=========================== PACKAGES =============================%
\usepackage[utf8]{inputenc}
\usepackage[left=0.5cm,right=0.5cm,vmargin=1.0cm]{geometry}

%--- LANGUE--%
\usepackage[brazilian]{babel}
%--- LANGUE--%

%--- FONT ----%
\usepackage[scaled]{helvet}
\renewcommand{\familydefault}{\sfdefault}
\usepackage{hyperref}
%--- FONT ----%

%--- PHOTO ----%
\usepackage{graphicx}
\usepackage{wrapfig}
%--- PHOTO ----%


{\renewcommand{\arraystretch}{1.3}}

\usepackage{titlesec} % Used to customize the \section command
\titleformat{\section}{\large \bf \uppercase}{}{0pt}{}[\titlerule] % Text formatting of sections
\titlespacing{\section}{17pt}{10pt}{5pt} % Spacing around sections

%=========================== PACKAGES =============================%


\author{Gustavo Ciotto Pinton}

\begin{document}

\pagestyle{empty}

%--- PUT YOUR PHOTO, IF NECESSARY --%
\begin{wrapfigure}{r}{3cm}
\vspace{-20pt}
\begin{center}
% \includegraphics[width=3cm, height=4cm]{23}
\end{center}
\end{wrapfigure}
%--- PHOTO --%

%======== PRENON ET NOM =========%
\textbf{\LARGE Gustavo CIOTTO PINTON}
%======== PRENON ET NOM =========%

%======== ADRESSE ===============%
Rua Voluntário Amador Lourenço, 57, Nova Itália, Valinhos

\textit{CEP 13271-393}, São Paulo, Brasil

(+55) 19 99898 5633, 23 anos

\url{gustavociotto@gmail.com}

\url{http://gciotto.github.io/portfolio}
%======== ADRESSE ===============%

\vspace{8pt}

%{\centerline {\large Candidatura às disciplinas de pós-graduação do IC como
%aluno especial }}

%{\centerline {Candidatura a um estágio de 6 a 12 meses com início em
%Janeiro/2016} }


%======== FORMATION ===============%
\section{Formação Acadêmica}

\begin{tabular}{p{.14\textwidth} p{.78\textwidth}}

\textbf{2016 - Presente} & \textbf{UNICAMP Universidade de Campinas}, Campinas,
Brasil.
\\
& Mestrado em Ciência de Computação - cursando disciplinas como aluno especial.
\vspace{4pt}\\

\textbf{2015 - 2016} & \textbf{UNICAMP Universidade de Campinas}, Campinas,
Brasil.
\\
\textbf{2011 - 2013} & Graduação em Engenharia de Computação - modalidade
Sistemas e Processos Industriais. Formado com distinção. \vspace{4pt}\\

\textbf{2013 - 2015} & \textbf{SUPÉLEC  École Supérieure d'Électricité},
Gif-Sur-Yvette, França. \\
& Duplo diploma em engenharia elétrica e de computação. \vspace{4pt} \\

\textbf{2008 - 2011} & \textbf{COTUCA Colégio Técnico de Campinas}, Campinas,
Brasil. \\
& Diploma técnico em Informática. \\
\end{tabular}

%======== FORMATION ===============%



%======== COMPETENCES ===============%
\section{Competências Linguísticas}

\begin{tabular}{p{.14\textwidth} p{.78\textwidth}}

\textbf{Português} & Língua materna.  \\

\textbf{Francês} & \textbf{562} pontos do \textbf{\textit{TCF}},
equivalente ao nível \textbf{C1} do CEFR. Teste realizado em \textit{Maio
de 2015}.\\

\textbf{Inglês} & \textbf{633} pontos do total de 677 do \textbf{\textit{TOEFL
ITP}}. Teste realizado em \textit{Março de 2015}. \\

\end{tabular}


\section{Competências Informáticas}

\begin{tabular}{p{.14\textwidth} p{.78\textwidth}}

\textbf{Sistemas} & Linux (Ubuntu, ArchLinux e Embarcado), Windows.  \\

\textbf{Linguagens} & C/C++, Python, Java, PHP,  SQL, \LaTeX.  \\

\textbf{Software} & Eclipse, \textit{Qt} Creator, Kinetis Design Studio,
Control System Studio, Matlab, MySQL Workbench, Microsoft SQL Server, Microsoft
Visual Studio\\

\textbf{Embarcados} & BeagleBone, FRDM-KL25Z. \\

\end{tabular}

%======== COMPETENCES ===============%


%======== EXPERIENCES ===============%
\section{Experiência Profissional}

\begin{tabular}{p{.14\textwidth} p{.78\textwidth}}


\textbf{Presente}  & Analista de Desenvolvimento, \textbf{Laboratório Nacional
de Luz Síncrotron - LNLS, Campinas, Brasil.}
\\
  & \vspace{-12pt}
  \begin{itemize}
    \item Gerenciamento de serviços EPICS. Desenvolvimento de Kernel Module Drivers para Linux embarcado.
	\end{itemize}\\

\textbf{02/2016 - 12/2016}  & Estagiário, \textbf{Laboratório Nacional
de Luz Síncrotron - LNLS, Campinas, Brasil.}
\\
& \vspace{-12pt}
\begin{itemize}
  \item Desenvolvimento de software e hardware para o sistema de controle da
  fonte de luz síncrotron \textit{Sirius}. Instalação dos sistemas de
  controle (gerenciador de alarmes, arquivamento e \textit{StreamDevice}) de variáveis EPICS. Desenvolvimento de
  \textit{Kernel Module Drivers} para a BeagleBone Black. Desenvolvimento
  de aplicações para \textit{Linux} embarcado.
  
\end{itemize}\\


\textbf{08/2015 - 12/2015}  & Assistente de laboratório, \textbf{FEEC
UNICAMP, Campinas, Brasil.}
\\
& \vspace{-12pt}
\begin{itemize}
  \item Assistência aos alunos nas práticas de laboratório da
  disciplina \textit{EA871 - Laboratório de Programação Básica de Sistemas
  Digitais}, envolvendo a programação dos recursos do \textit{kit} FRDM-KL25Z
  da NXP.
  
\end{itemize}\\

%  \textbf{07/2014 - 09/2014}   & Estagiário em  produção,
%  \textbf{Elis -  usina de Brétigny-sur-Orge, França.}\\
%   & \vspace{-12pt}
%   \begin{itemize}
%     \item Preparação dos pedidos dos clientes. \vspace{-8pt}
%     \item Ordenação dos produtos e operação das máquinas.
%   \end{itemize} \\

\textbf{02/2013 - 07/2013}   & Assistente de laboratório, \textbf{IC
UNICAMP - Instituto de Computação, Campinas, Brasil.} \\
& \vspace{-12pt}
\begin{itemize}
  \item Assistência aos alunos de primeiro ano nas práticas de laboratório da
  disciplina \textit{MC102 - Algoritmos e Programação de Computadores}.
  \vspace{-8pt}
  \item Elaboração e concepção de ferramentas pedagógicas em C
  abordando alguns assuntos fundamentais tais como recursividade, ponteiros e
  algoritmos de procura e ordenação.
\end{itemize}\\

\textbf{08/2011 - 06/2012} & Assistente de pesquisa, \textbf{IC UNICAMP -
Instituto de Computação, Campinas, Brasil}. \\
& \vspace{-12pt}
\begin{itemize}
  \item Análise da performance de softwares distribuídos em toda a rede em
  função des sistemas operacionais, da eficiência da Java Virtual Machine (JVM)
  e das condições de hardware.
\end{itemize}
\\


\textbf{12/2010 - 02/2011} & Estagiário técnico - \textbf{Movile,
Campinas, Brasil.}
\\  &  \vspace{-12pt}
\begin{itemize} 
\item Responsável pela manutenção do banco de dados (MySQL Server) e de
alguns sistemas web (PHP e JavaScript) da empresa.
\end{itemize}
\end{tabular}
\vspace{-12pt}

%======== PUBLICATIONS ===============%

\section{Publicações}

\begin{tabular}{p{.14\textwidth} p{.78\textwidth}}

\textbf{07/2016}  & \textit{S. Lescano, A. R. D. Rodrigues, E. P.
Coelho, G. C. Pinton, J. G. R. S. Franco e P. H. Nallin}. \textbf{UVX
control system: An approach with Beaglebone Black}. Personal Computers and
Particle Accelerator Controls, 2016.
\\

\end{tabular}

\section{Premiações}

\begin{tabular}{p{.14\textwidth} p{.78\textwidth}}

\textbf{06/2013}  & \textbf{Vencedor da Bolsa de Estudos Eiffel, Campus
France}. \vspace{4pt} \\

\textbf{12/2016}  & \textbf{Prêmio Intituto de Engenharia, Instituto de Engenharia}.\\
& Melhor aluno da turma de formandos de 2016 da Faculdade de Engenharia Elétrica
e de Computação - UNICAMP Engenharia de Computação - Sistemas e Processos Industriais. \vspace{4pt}\\

\textbf{12/2016}  & \textbf{Certificado de Honra ao Mérito, CREA-SP}. \\
& Melhor aluno da turma de formandos de 2016 da Faculdade de Engenharia Elétrica
e de Computação - UNICAMP Engenharia de Computação - Sistemas e Processos Industriais. \vspace{4pt}\\

\end{tabular}

%======== ACTIVITES ===============%
% \section{Atividades Diversas}
% 
% \begin{tabular}{p{.14\textwidth} p{.77\textwidth}}
% 
% \textbf{Sport} & Tênis e futebol. \\
% 
% \end{tabular}

%======== ACTIVITES ===============%

\end{document}
